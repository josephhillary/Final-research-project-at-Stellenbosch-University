\chapter{Conclusion \label{ch5}}
This dissertation aimed to research on Lie group analysis of nonlinear partial differential equations (NLPDEs). It must be said that NLPDEs are very important in describing various phenomena and systems in real life. Our main concern in this study was the nonlinear coupled system of Korteweg-de Vries (KdV) equations that describe the dynamics of solitons. However, this powerful approach that uses invariant properties of an NLPDE to get exact solutions and conservation laws can be used for any NLPDE.

The first step was to develop mathematical concepts of Lie group analysis and conservation laws in Chapter \ref{ch2}. We developed pertinent notions of local Lie groups, prolongations, symmetry groups and invariants. Also introduced is the concept of conservation laws and two methods to compute them, that is, the method of multipliers and Ibragimov's conservation theorem.

Thereafter, we studied the KdV equation as an illustrative example in Chapter \ref{ch3}. This involved finding its space and time translations, Galilean boost, and scaling symmetries. We then performed symmetry reductions that resulted in a total of four group-invariant solutions. A linear combination of time and space translations also helped in finding the one-soliton solution for the KdV equation. Finally, we found conservation laws for mass, momentum, and energy by using the multipliers method.


Our third step was to investigate a system of coupled KdV equations in Chapter \ref{ch4}. Likewise to the KdV equation studied in Chapter \ref{ch3}, a four-dimensional Lie algebra of symmetries was found for the nonlinear coupled system KdV equations. This was also spanned by  space and time translations, Galilean boost and scaling symmetries where the scaling symmetry acts on four variables. Associated to each symmetry, we obtained symmetry reductions that gave six nontrivial solutions for the coupled system.  Lastly, we constructed infinite conservation laws of a nonlinear coupled KdV system by using multipliers and a theorem proposed by Nail Ibragimov. Three of these laws show that mass, momentum and energy are conserved quantities in the evolution of a nonlinear coupled KdV system.

The above results show a very interesting property of the KdV equation. Most important to note is that the infinite number of conservation laws for the coupled system show that the KdV equation is completely integrable, meaning that the behavior of the system can be determined by initial conditions and can be integrated from the prescribed initial conditions. Indeed, the KdV equation gives rise to multiple-soliton solutions thus emphasizing the importance of the KdV equation in the theory of integrable systems. The beautiful KdV equation is ubiquitous, having applications in various settings.
The equation has been used to describe the dynamics of solitons, ion-acoustic waves in a plasma, shallow-water waves  and 
nonlinear perturbations along internal surfaces between layers of different densities in stratified fluids, for example propagation of solitons of long internal waves in oceans. The KdV equation also models shock wave formation, turbulence, boundary layer behavior, and mass transport.


\section*{Further work}
 For future work, we are interested in applying the conservation laws to construct exact solutions to the nonlinear coupled system of Korteweg-de Vries equations.