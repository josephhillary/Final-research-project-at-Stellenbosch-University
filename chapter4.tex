\chapter{Nonlinear Coupled  Korteweg-de Vries (KdV) Equations \label{ch4}}
We have demonstrated in Chapter \ref{ch3}, how to find Lie point symmetries, symmetry reductions, and invariant solutions for a nonlinear PDE. As an illustrative example,
we studied the nonlinear Korteweg-de Vries equation (\ref{kd}), for which we found invariant solutions which included a one-soliton solution. Conservation laws for mass, momentum, and energy were also constructed by using multipliers. This chapter will focus on Lie group analysis a nonlinear coupled system of Korteweg-de Vries equations. Our aim is to perform a Lie group analysis of this system. This we achieve by calculating Lie point symmetries, performing symmetry reductions, and constructing invariant solutions for the nonlinear coupled KdV system. Thereafter,  we find conservation laws by use of both the method of multipliers and a theorem due to Ibragimov.
\section{Introduction}
Consider the Kortweg-de Vries equation  \begin{equation} q_t + \alpha q q_x + \beta q_{xxx} =0, \label{cpkdv}
\end{equation} $ \alpha$ and $ \beta$ are constants.
Let \begin{equation}
q(t,x) = u(t,x)+ i v(t,x),\label{qcmplx}
\end{equation} where $ i^2=-1$. Then substituting (\ref{qcmplx}) into (\ref{cpkdv}) and separating the real and imaginary parts, we obtain \begin{align} \label{real} \Delta_1 &\equiv u_t + \alpha u u_x - \alpha v v_x + \beta u_{xxx} =0,\\
 \label{imgry} \Delta_2 & \equiv v_t + \alpha u v_x + \alpha v u_x + \beta v_{xxx}=0,
\end{align} which is a nonlinear system of coupled KdV equations. We perform Lie symmetry analysis on (\ref{real})-(\ref{imgry}), that is , we obtain Lie point symmetries,  invariant solutions and conservation laws of (\ref{real})-(\ref{imgry}).
\section{Solutions of the nonlinear coupled KdV Equations (\ref{real})-(\ref{imgry})}
\subsection{Lie point symmetries of (\ref{real})-(\ref{imgry}) \label{liepoitsymts}}
To find Lie point symmetries of the coupled KdV Equations, we use Lie's method. The infinitesimal transformations of the Lie group with parameter $\epsilon$ are 
\begin{align}
\begin{aligned}\bar{t}=t+  \xi^t(t,x,u,v)\epsilon,\,\, \bar{x}=x+  \xi^x(t,x,u,v)\epsilon,\,\,
  \bar{u}=u+  \eta^u(t,x,u,v)\epsilon,\,\, 
  \bar{v}=v+  \eta^v(t,x,u,v)\epsilon.
\label{infintli} 
\end{aligned}
\end{align}
The vector field \begin{equation}
	X= \xi^t(t,x,u,v) \frac{ \partial }{ \partial t}+ \xi^x (t,x,u,v)\frac{ \partial }{ \partial x}+ \eta^u(t,x,u,v) \frac{\partial}{ \partial u } + \eta^v(t,x,u,v) \frac{ \partial }{ \partial v},
\end{equation} is a Lie point symmetry of (\ref{real})-(\ref{imgry}) if \begin{align}\label{Eq1} \mbox{X}^{[3]}  \Delta_1\Big |_{\Delta_1=0,  \,\,\Delta_2=0} =0, \\ \label{Eq2}\mbox{X}^{[3]}  \Delta_2 \Big|_{\Delta_1=0, \,\, \Delta_2=0} =0.
\end{align} 
Expanding  (\ref{Eq1}) and (\ref{Eq2}) and splitting on derivatives of $v$ and $u$, we have an overdetermined system of ten  PDEs, namely,
\begin{align}\begin{aligned}\xi^t_u =0, \quad \xi^t_v =&0, \quad \xi^t_x =0,\quad \xi^x_u =0,\quad
 \xi^x_v =0,\quad  \xi^t_{tt}=0,\quad \xi^x_{tt}=0, \quad 3 \xi^x_x - \xi^t_t =0,\\
 3 \eta^v + 2 \xi^t_t v=&0,\quad 
  3 \alpha \eta^u + 2 \alpha \label{cp10}\xi^t_t u -3\xi^x_t =0.\end{aligned}
\end{align}
Solving the system (\ref{cp10}) in the same way we did for (\ref{1})-(\ref{8}), we obtain 
\begin{align}
\label{xit} \xi^t = & A_1+3 A_2t,\\
\label{xix} \xi^x =&A_2 x + \alpha  A_3 t + A_4,\\
\label{etau} \eta^u =&  - 2A_2 u+ A_3,\\
\label{etav} \eta^v =& -2A_2 v,
\end{align} for arbitrary constants $ A_1, A_2,A_3,A_4$.  Hence from  (\ref{xit})-(\ref{etav}),
the infinitesimal  symmetries of the coupled KdV Equations (\ref{real})-(\ref{imgry}) is a Lie algebra generated by the vector fields
\begin{align}
X_1 = &\frac{ \partial}{ \partial t}, \\
X_2 =& \frac{ \partial}{ \partial x},\\
X_3 = & \alpha t \frac{ \partial }{ \partial x}+ \frac{\partial }{ \partial u},\\
X_4 =& 3t \frac{ \partial }{ \partial t} + x\frac{ \partial}{ \partial x} - 2 u \frac{ \partial }{ \partial u}-2 v \frac{ \partial }{ \partial v}.
\end{align}
 \subsection{Commutator table}
 The set of all infinitesimal symmetries of coupled KdV equations forms a Lie algebra. By the definition of a commutator in Equation (\ref{sec2.7.2}), we get the commutation relations, which we present in Table \ref{Tab:3}.
 \begin{center}
 	
 	\begin{tabular}{|c |c|c|c|c|} 
 		\hline 
 		$[X_i,X_j]$	& $ X_1$ & $ X_2$ & $ X_3$ & $ X_4$ \\ 
 		\hline 
 		$ X_1$ & 0 & 0 & $\alpha X_2$ & $ 3X_1$ \\ 
 		\hline 
 		$ X_2$ & 0 & 0 & 0 & $ X_2$ \\ 
 		\hline 
 		$ X_3$ & -$\alpha X_2$ & 0 & 0 & -2$X_3$\\ 
 		\hline 
 		$ X_4$ & -3$X_1$ & -$ X_2$ &2$X_3$  & 0 \\ 
 		\hline 
 	\end{tabular}\captionof{table}{ \label{Tab:3} A commutator table for the Lie algebra generated by the symmetries of coupled KdV equation.}
 \end{center}
\subsection{Local Lie groups}
In this subsection, we present the corresponding group of transformations relating to each Lie point symmetry of coupled KdV system. We obtain the groups by invoking the Lie equations as defined in Theorem \ref{lieeqns}. For the system (\ref{real})-(\ref{imgry}), we have the Lie equations, for $i=1,2,3,4,$
\begin{align} \begin{aligned}
 \frac{ \mathrm{d} \bar{t}}{ \mathrm{d} \epsilon_i} = \xi^t( \bar{t},\bar{x},\bar{u},\bar{v}),\quad \bar{t}\Big |_{ \epsilon_i =0} =t, \quad
\frac{ \mathrm{d} \bar{x}}{  \mathrm{d} \epsilon_i} = \xi^x( \bar{t},\bar{x},\bar{u},\bar{v}),\quad \bar{x}\Big |_{ \epsilon_i =0} =x, \\
\frac{ \mathrm{d} \bar{u}}{  \mathrm{d} \epsilon_i} = \eta^u( \bar{t},\bar{x},\bar{u},\bar{v}),\quad \bar{u}\Big |_{ \epsilon_i =0} =u, \quad \frac{ \mathrm{d} \bar{v}}{  \mathrm{d} \epsilon_i} =\eta^v( \bar{t},\bar{x},\bar{u},\bar{v}),\quad \bar{v}\Big |_{ \epsilon_i =0} =v. \label{liq} \end{aligned}
\end{align}
 Taking for each $X_i, i =1,2,3,4, \,\,  T_{\epsilon_i}$ to be the corresponding group of transformations, we obtain from the Lie equations (\ref{liq}) the groups;
 \begin{align}
  T_{\epsilon_1}& :  \bar{t}=t+\epsilon_1, \,\, \bar{x}=x,\,\,  \bar{u}=u,\,\, \bar{v}=v,\\
  T_{\epsilon_2}& :  \bar{t}=t, \,\,  \bar{x}=x +\epsilon_2 ,\,\, \bar{u}=u, \,\, \bar{v}=v,\\ 
 T_{\epsilon_3}& :  \bar{t}=t, \,\, \bar{x}=x + \alpha\epsilon_3t,\,\, \bar{u}=u+\epsilon_3,\,\, \bar{v}=v,\\
  T_{\epsilon_4} &:  \bar{t}=t e^{3 \epsilon_4}, \,\, \bar{x}= xe^{\epsilon_4},\,\, \bar{u} = ue^{-2\epsilon_4},\,\, \bar{v}=ve^{-2\epsilon_4}. \label{lclg4}
 \end{align}
\subsection{Symmetry reductions of the coupled KdV Equations (\ref{real})-(\ref{imgry})}
We use the symmetries obtained in \textbf{Subsection \ref{liepoitsymts} Lie point symmetries of (\ref{real})-(\ref{imgry})} to perform symmetry reductions for the Coupled KdV Equations (\ref{real})-(\ref{imgry}).
\begin{enumerate} [(i)]
\item  \textbf{The  time translation symmetry $X_1 = { \partial } /{ \partial t}$. \label{stat}} \newline
Solving the characteristic equations associated to the operator $X_1$ gives the  invariants \newline$J_1 =x, \quad  J_2 = u,\quad \text{and} \quad  J_3 =v.$
 Hence, we have
 \begin{align}
 \label{gp1}	u =\varphi(x), \quad 	 v= \psi(x), 
 \end{align} for arbitrary functions $\varphi $ and  $\psi $. Substituting  the expressions  for $u$
and $v$ given by (\ref{gp1}) into the system (\ref{real})-(\ref{imgry}), we get a system of third order ordinary DEs namely, 
\begin{align} \alpha \left[  \label{krp}\varphi(x) \varphi'(x)- \psi(x) \psi'(x) \right] + \beta \varphi'''(x)=0,\\
\alpha \left( \varphi(x) \psi(x) \right)' + \beta \psi'''(x)=0. \label{spt}\end{align}
Integration of the system (\ref{krp})-(\ref{spt}) yields;
\begin{align}
\label{a} \frac{\alpha}{2}\left[  \varphi(x)^2- \psi(x)^2  \right] + \beta \varphi''(x)=C_1,\\
\label{b} \alpha \left[ \varphi(x) \psi(x)\right] + \beta \psi''(x)=C_2,
\end{align} for arbitrary constants $C_1$ and $C_2$.
If we take $C_1=C_2=0$, the system (\ref{a})-(\ref{b}) becomes   \begin{align}
\label{a1} \frac{\alpha}{2}\left[  \varphi(x)^2- \psi(x)^2  \right] + \beta \varphi''(x)=0,\\
\label{b1} \alpha \left[ \varphi(x) \psi(x)\right] + \beta \psi''(x)=0,
\end{align}

To find more solutions of the system (\ref{a1})-(\ref{b1}), we determine its Lie point symmetries.
Using the Lie's algorithm for computing point symmetries, we see that  the Lie point symmetries of (\ref{a1})-(\ref{b1}) are \begin{align} X_1^{*} = \frac{\partial }{ \partial x},\,\, X_2^{*}=x \frac{ \partial }{ \partial x}- 2 \varphi \frac{ \partial }{ \partial \varphi}-2 \psi
\frac{\partial }{ \partial \psi}.
\end{align}
Proceeding as above, we see that the symmetry $X_1^{*}$ yields the trivial solution \begin{align}
u=0,\,\, v=0. \label{tri}
\end{align}
The second symmetry $ X_2^{*}$  has the characteristic equations \begin{align}
\frac{\mathrm{d}x}{x}= \frac{\mathrm{d} \varphi}{-2 \varphi}=\frac{\mathrm{d} \psi }{-2 \psi},
\end{align} 
which provides the invariants $ J_1 = x^2 \varphi, \,\,  J_2 = x^2 \psi$.
Letting $\varphi =  {\lambda}/{x^2}, \,\,  \psi =  {\mu}/{x^2}$,  substituting the values of $\varphi$ and $\psi$ into (\ref{a1})-(\ref{b1}) and solving the resulting equations yield: 
\begin{enumerate}[(a)]
	
\item \textbf{Case one.} Taking 
 $ \mu =0$ gives $\lambda =0$ or $ \lambda = -12\beta/ \alpha$.\newline
 The case $ \lambda = 0,\,\, \mu=0$ also gives the trivial solution (\ref{tri}).
One can easily see that  $ \lambda = -12\beta/\alpha,\,\, \mu=0$ gives $\varphi=- {12 \beta}/(\alpha x^2)$, 
$\psi=0$ which  is a solution of the system (\ref{a1})-(\ref{b1}).
Hence
\begin{align}
u_1(t,x)=-\frac{12 \beta}{ \alpha x^2}, \,\, v_1(t,x)=0,
\end{align} 
is a solution of the coupled KdV system (\ref{real})-(\ref{imgry}).

\item \textbf{Case two.} Taking
$\lambda=- {6\beta}/{\alpha}, \,\, \mu = \pm   {6 \beta i}/{\alpha}$ with $i^2=-1$.  Consequently, 
\begin{align}
u_2(t,x) = - \frac{6  \beta }{ \alpha x^2},\,\,
v_2(t,x) =  \frac{ 6 i \beta }{ \alpha x^2} 
\end{align} 
and 
\begin{align}
u_3(t,x) =- \frac{6  \beta }{ \alpha x^2}, \,\,
v_3(t,x) =  -\frac{6 i \beta }{ \alpha x^2} 
\end{align} 
are solutions of the coupled KdV system (\ref{real})-(\ref{imgry}). Hence Lie group analysis has given us three steady-state solutions for the coupled KdV system (\ref{real})-(\ref{imgry}) under the time translation symmetry $X_1 = {\partial }/{ \partial t}$.

\end{enumerate}
\item \textbf{The space  translation symmetry $X_2 = {\partial }/{\partial x}.$}

Solving the characteristic equations  associated to $X_2$ gives the invariants $J_1=t, \quad J_2 = u \quad  \text{and} \newline \quad J_3 =v.$ Therefore, the group-invariant solution is
\begin{align}
u =\phi(t), \quad  v= h(t), \label{gp2}	
\end{align}  for arbitrary functions $h$ and $\phi$. Substitution of the solutions from (\ref{gp2}) into  (\ref{real})-(\ref{imgry}), we get a system of first order ordinary DEs, namely,
\begin{equation}\phi'(t)=0, \quad h'(t)=0,
\end{equation} which is integrated once with respect to $t$ to  yield,
\begin{align}
\phi(t)=C_1, \quad h(t)=C_2,
\end{align} for arbitrary constants $C_1$ and $C_2$. Consequently, the space translation group-invariant solution  of the system (\ref{real})-(\ref{imgry}) is
\begin{align}
u(t,x)=C_1,  \quad  v(t,x)=C_2.
\end{align} 
\item  \textbf{The Galilean boost symmetry} $X_3 =  \alpha t { \partial }/{ \partial x}+{\partial }/{ \partial u}.$

Solving the characteristic equations associated to Galilean boost gives  the invariants
\begin{align}
J_1 =t, \quad  J_2 =v,   \quad  J_3 = -u + \frac{x}{ \alpha t}, \,\, t \neq 0.\label{onyan}
\end{align} 

Thus  the invariant solution of (\ref{real})-(\ref{imgry}) is
\begin{align} u = \frac{x}{ \alpha t}-g(t),  \,\,   v = f(t), \,\, t\neq 0, \label{gpi3}
\end{align}  for arbitrary functions $f$ and $g$.
Substitution of the values of $u$ and $v$ from (\ref{gpi3}) into the System (\ref{real})-(\ref{imgry}), we get a nonlinear system of coupled  first order ordinary DEs, namely,
\begin{align} tg'(t)+g(t)=0,\\
tf'(t)+ f(t) =0,
\end{align} whose solutions are $ g(t) = {C_1}/{t}$ and $f(t)= {C_2}/{t} $ for arbitrary constants $ C_1$ and $C_2$.
Hence the Galilean boost group-invariant solution of the system (\ref{real})-(\ref{imgry}) is
 \begin{align}
 u(t,x) =\frac{x+A}{\alpha t}, \quad  v(t,x)= \frac{C_2}{t},
\end{align} where $ A=-\alpha C_1$ and $ t \neq 0. $

\item \textbf{The scaling  $X_4 =3t \frac{ \partial }{ \partial t} + x\frac{ \partial}{ \partial x} - 2 u \frac{ \partial }{ \partial u}-2 v \frac{ \partial }{ \partial v}$.
} \newline
By solving of the characteristic equations associated to this symmetry, we obtain the invariants 
\begin{align} J_1 = \frac{x^3}{t}, \quad  J_2 =ux^2,  \quad  J_3 = vx^2.
\end{align}
Generally, the group-invariant solution pair is 
\begin{align}
	u(t,x)=  \frac{f(\lambda) }{x^2},\quad  v(t,x)= \frac{g(\lambda) }{x^2}, \quad  \lambda=\frac{x^3}{t},
\end{align} where  the functions $ f$ and $g$ satisfy  the system of third order nonlinear  coupled  ordinary DEs
\begin{align}
2 \alpha (g^2-f^2)- \lambda^2 f'  +3 \alpha \lambda (ff'-gg') + \beta (-24f + 24 \lambda f'  + 27\lambda^3 f''' )=&0,\\
-4\alpha fg-  \lambda^2 g'  + 3 \alpha \lambda(fg)' + \beta (-24g + 24 \lambda g' +27 \lambda^3g''' )=&0.
\end{align} 
\item \textbf{Linear combination of time and space translations	$X_1 + c X_2$}. \newline
We consider a symmetry $X$, which is a linear combination of the time and space translations symmetries,  that is, 
$
X= \partial_t+ c \partial_x,
$ for a constant $c$.
The invariants associated to this symmetry $X$ are 
$
J_1 = x-ct, \,\, J_2 =u, \,\,   J_3 =v.
$
Hence, the invariant solution for the symmetry  $X$ is 
\begin{align} \label{wav1} 
u = f(x-ct), \,\, v=g(x-ct), 
\end{align}  
for arbitrary functions $f$ and $g$. Substitution of $u$ and $v$ from  (\ref{wav1}) into the system (\ref{real})-(\ref{imgry}) yields a system of nonlinear third order ordinary DEs, namely \begin{align}
-cf' (\xi)+ \alpha \left\lbrace f (\xi) f' (\xi)- g (\xi) g' (\xi) \right\rbrace  + \beta f''' (\xi)=0,\\
-cg' (\xi)+ \alpha ( f (\xi)g (\xi))'+ \beta g'''(\xi)=0,
\end{align} 
which on integrating once with respect to $\xi$ yields
\begin{align}
 \label{tr1}
 -c f + \frac{1}{2} \alpha (f^2-g^2) + \beta f''+C_1=0, \\
 \label{tr2} -cg+ \alpha f g + \beta g''+ C_2 =0,
\end{align} 
for arbitrary constants $ C_1$ and $C_2$.
\begin{rem} If we take the constants $ C_1=C_2=0$, then when the wave velocity $c=0$, we can recover the stationary solutions given in (\ref{stat}).
\end{rem}
\begin{rem} Traveling wave solutions of the system (\ref{real})-(\ref{imgry}) must satisfy the system (\ref{tr1})-(\ref{tr2}).	
\end{rem}
\end{enumerate}
\section{Conservation laws of the coupled KdV Equations (\ref{real})-(\ref{imgry})}
Computation of conservation laws for the coupled KdV Equations (\ref{real})-(\ref{imgry}) is done using two methods; the method of  multipliers  and a theorem due to Ibragimov.
\subsection{Conservation laws of (\ref{real})-(\ref{imgry}) using the multipliers}
We seek local conservation law multipliers for the system (\ref{real})-(\ref{imgry}), whose determining equations are 
\begin{align}
\label{vau}\frac{\delta}{ \delta u} \left[ \Lambda^1\Delta_1 + \Lambda^2 \Delta_2 \right] =0,\\
\label{vav}\frac{\delta}{ \delta v} \left[ \Lambda^1\Delta_1 + \Lambda^2 \Delta_2 \right] =0,
\end{align} where 
\begin{align} \frac{ \delta }{ \delta u} &= \frac{ \partial }{ \partial u}-D_t \frac{\partial }{ \partial u_t}-D_x \frac{\partial }{ \partial u_x}+ D^2_x \frac{ \partial }{ \partial u_{xx}}-D_x^3 \frac{ \partial }{ \partial u_{xxx}}+ \ldots,\\
\frac{ \delta }{ \delta v} &= \frac{ \partial }{ \partial v}-D_t \frac{\partial }{ \partial v_t}-D_x \frac{\partial }{ \partial v_x}+ D^2_x \frac{ \partial }{ \partial v_{xx}}- D_x^3\frac{ \partial }{ \partial v_{xxx}}+ \cdots,
\end{align} are the Euler-Lagrange operators and \begin{align}
\label{dt} D_t =& \frac{\partial}{\partial t} + u_t\frac{\partial}{\partial u}+ v_t\frac{\partial}{\partial v}
+ u_{tx} \frac{\partial}{\partial u_x} +  v_{tx} \frac{\partial}{\partial v_x}+ u_{tt}\frac{\partial}{\partial u_t} + v_{tt}\frac{\partial}{\partial v_t} + \cdots  , \\
\label{dx} D_x =& \frac{\partial}{\partial x} + u_x \frac{\partial}{\partial u}+ v_x \frac{\partial}{\partial v}+ u_{xx}\frac{\partial}
{\partial u_x} + v_{xx}\frac{\partial}
{\partial v_x}+ u_{tx} \frac{\partial}{\partial u_{t}} + v_{tx} \frac{\partial}{\partial v_{t}}+\cdots,
\end{align} are total derivatives operators.
We look for second order multipliers,  that is, 
\begin{align}
	\Lambda^n = \Lambda^n(t,x,u,u_x,u_{xx},v,v_x,v_{xx}), \quad n = 1,2.
\end{align}
The determining Equations  (\ref{vau})-(\ref{vav}) become
 \begin{align}
 \label{det1}\frac{\delta}{ \delta u} \left[ \Lambda^1 \{  u_t + \alpha u u_x - \alpha v v_x + \beta u_{xxx}   \}   + \Lambda^2 \{   v_t + \alpha u v_x + \alpha v u_x + \beta v_{xxx} \} \right] =0,\\
 \label{det2}\frac{\delta}{ \delta v} \left[ \Lambda^1 \{  u_t + \alpha u u_x - \alpha v v_x + \beta u_{xxx} \} + \Lambda^2 \{ v_t + \alpha u v_x + \alpha v u_x + \beta v_{xxx}  \} \right] =0.
 \end{align} 
 Expanding  (\ref{det1})-(\ref{det2})  and splitting on derivatives of $u$ and $v$ yields an overdetermined system of 22 PDEs, namely \begin{align} \begin{aligned} &\Lambda_{xx}^1=0, \,\,  \Lambda_{xx}^2=0, \,\, \Lambda_{vx}^1=0, \,\, \Lambda_{vx}^2=0, \,\,
 \Lambda_{xv_{xx}}^1=0, \,\,  \Lambda_{xv_{xx}}^2=0, \,\,  \beta \Lambda_{vv}^1-\alpha \Lambda_{v_{xx}}^2=0,
 \\
 & \beta \Lambda_{vv}^2+ \alpha \Lambda_{vv_{xx}}^1=0, \,\,
 \Lambda_{vv_{xx}}^1=0, \quad \Lambda^2_{vv_{xx}}=0, \,\, \Lambda_{v_{xx}v_{xx}}^1=0, \,\,
 \Lambda_{v_{xx}v_{xx}}^2=0, \,\,
 \Lambda^{1}_u+\Lambda^{2}_v=0, 
 \\ 
 &   \Lambda_{t}^1+ \alpha \left(\Lambda_{x}^2v +\Lambda_{x}^1u\right)=0,\,\,
 \Lambda_{t}^2+ \alpha \left( \Lambda_{x}^2u- \Lambda_{x}^1v\right)=0,  \,\,  \Lambda^{2}_u- \Lambda^{1}_v=0, \,\,\Lambda^{1}_{u_x}=0,
 \\ 
 &   
 \quad \Lambda^{2}_{u_x}=0, \,\, \Lambda^{1}_{u_{xx}}+ \Lambda^{2}_{v_{xx}}=0,  
 \Lambda^{2}_{u_{xx}}-\Lambda^{1}_{v_{xx}}=0, \,\, \Lambda^{2}_{v_{x}} =0 ,\,\, \Lambda^{1}_{v_{x}}=0 .
 \label{od1}
 \end{aligned}\end{align}
Calculations reveal the solution of the system (\ref{od1})
as  \begin{align} \begin{aligned}
\Lambda^1 =& \frac{\alpha }{ 2 \beta} \left( c_3 \{ u^2-v^2\} + 2c_4uv \right) 
+ (c_2t+c_5)u + (c_1t+c_6)v+ c_3 u_{xx}\\& + c_4v_{xx}  +c_7  -\frac{1}{\alpha} c_{2} x,\\
\Lambda^{2}=& \frac{\alpha }{2 \beta } \left(c_4 \{u^2-v^2\}  -2 c_3uv+  \right) +(c_1t + c_6)u-(c_2t +c_5)v +c_4 u_{xx} \\&-c_3 v_{xx}+c_8   - \frac{1}{ \alpha}c_1  x,\end{aligned} \end{align} for arbitrary constants $c_1,\ldots,c_8$.
\begin{rem}
Essentially, the nonlinear coupled system of KdV Equations (\ref{real})-(\ref{imgry}) have eight sets of local conservation law multipliers.
\end{rem}
\begin{rem} Recall that the multipliers must satisfy the property defined in (\ref{sec2.9.13}). For the nonlinear coupled KdV system  (\ref{real})-(\ref{imgry}), we have
	\begin{align}
	\label{oda}		\Lambda^1  \Delta_1+ \Lambda^2 \Delta_2 = D_t T^t + D_x T^x,
	\end{align} where 
	$T^t=T^t(t,x,u,v)$	and $T^x=T^x(t,x,u,v,u_x,v_x,u_{xx},v_{xx}).$	
\end{rem}
Solving (\ref{oda}) in the same way we did in  (\ref{consl}) for the KdV Equation (\ref{kd}), we obtain conserved vectors  corresponding to each set of multipliers as shown below.
\begin{enumerate}[(i)]
	\item The multiplier $ \left( \Lambda_{1}^1, \Lambda_{1}^2\right)= \left( tv,tu-\frac{x}{ \alpha}\right)$ has the  conserved vectors 
\begin{align}
T_1^t=& tuv-\frac{xv}{\alpha},\\
T_1^x=& \beta \left[ t\{vu_{xx} +uv_{xx} -v_x u_x \}   + \frac{1}{\alpha}\{v_x-xv_{xx}\} \right]+ \alpha \left[ t \left( u^2v-\frac{v^3}{3}\right) \right] -xuv.
\end{align}

\item The multiplier $ \left( \Lambda_{2}^1, \Lambda_{2}^2\right)= \left( tu-\frac{x}{ \alpha},-tv\right)$ has the  conserved vectors 
\begin{align} \begin{aligned}
T_2^t=& \frac{ t}{2} \{u^2-v^2  \} -\frac{xu}{ \alpha},\\
T_2^x=& \beta\left[t \left( uu_{xx}-vv_{xx} + \frac{ 1}{2} \{v_x^2-u_x^2 \} \right) +  \frac{1}{\alpha}\{u_x-xu_{xx} \}   \right] + \alpha t \left[ \frac{u^3}{3} -uv^2 \right]\\ & + \frac{x}{2} \{v^2-u^2 \}.\end{aligned}
\end{align}

\item  The multiplier $ \left( \Lambda_{3}^1, \Lambda_{3}^2\right)= \left(  \frac{ \alpha }{2 \beta} \{u^2- v^2 \}+ u_{xx},  -\{ \frac{ \alpha uv}{\beta} +v_{xx}\} \right)$ has the  conserved vectors 
\begin{align}
T_3^t=& \frac{\alpha }{2 \beta}
\left( \frac{u^3}{3}-uv^2\right),\\
T_3^x=& \frac{\alpha}{2} \left[ (u^2-v^2)u_{xx} -v^2v_{xx} \right] - \alpha u v v_{xx}+ \frac{\beta}{2} \left[ u_{xx}^2-v_{xx}^2\right]+ u_tu_x-v_tv_x \\& + \frac{ \alpha^2}{4 \beta} \left[ \frac{1}{2} \{ u^4+v^4 \}-3u^2v^2 \right].
\end{align}
\item  The multiplier $ \left( \Lambda_{4}^1, \Lambda_{4}^2\right)= \left( \{ \frac{ \alpha uv}{\beta} +v_{xx}\} ,  \frac{ \alpha [u^2- v^2 ] }{2 \beta}+ u_{xx}\right)$ has the  conserved vectors 
\begin{align}
T_4^t=& \frac{\alpha }{2 \beta}
\left(u^2v - \frac{v^3}{3}\right),\\
T_4^x=& \frac{ \alpha^2}{ 2 \beta} \left[  ( u^3v-uv^3)\right]+ v_tu_x + u_tv_x+ \frac{\alpha}{2} (u^2-v^2)v_{xx} + \{\alpha uv + \beta v_{xx} \}u_{xx}.
\end{align}

\item  The multiplier $ \left( \Lambda_{5}^1, \Lambda_{5}^2\right)= \left(u , -v\right)$ has the  conserved vectors 
\begin{align}
 T^t_5= \frac{1}{2}\{ u^2-v^2\},\quad
 T^x_5=\beta \left( u u_{xx} -v v_{xx} + \frac{ v_x^2-u_x^2}{2} \right)  +  \alpha \left(\frac{u^3}{3} - uv^2 \right).
\end{align}

\item  The multiplier $ \left( \Lambda_{6}^1, \Lambda_{6}^2\right)= \left(v, u\right)$ has the  conserved vectors 
\begin{align}
 T^t_6= uv,\quad
T^x_6= \beta \left( v u_{xx} +u v_{xx} -u_x v_x \right) + \alpha \left( u^2 v-  \frac{  v^3}{3} \right).
\end{align}
 \item The multiplier  $  \left( \Lambda_{7}^1, \Lambda_{7}^2\right)= \left( 1,0\right)$
 has the  conserved vectors 
 \begin{align}
 T_7^t=u,\quad
 T_7^x= \frac{\alpha}{2} \{ u^2-v^2\}+ \beta u_{xx}.
 \end{align}
\item The multiplier has  $ \left( \Lambda_{8}^1, \Lambda_{8}^2\right)= \left( 0,1\right) $ the conserved vectors 
\begin{align}
T_8^t=v,\quad
T_8^x=\alpha uv+ \beta v_{xx}.
\end{align}
\end{enumerate}
\begin{rem}
It can be verified that \begin{align}
D_t T^t_i + D_xT^x_i \Big |_{\Delta_1=0, \,\,\Delta_2=0}=0, \label{bao}
\end{align} for $i =1,\ldots,8$.
\end{rem}
\begin{rem}
The expressions in (\ref{bao}) are  eight conservation laws for the coupled KdV system (\ref{real})-(\ref{imgry}).
\end{rem}
\begin{rem}
The presence of multipliers $\left( \Lambda_{7}^1, \Lambda_{7}^2\right)= \left( 1,0\right)$ and $\left( \Lambda_{8}^1, \Lambda_{8}^2\right)= \left( 0,1\right)$ manifest that the coupled KdV equations are themselves conservation laws.
\end{rem}
\subsection{ Conservation laws of (\ref{real})-(\ref{imgry}) using Ibragimov's theorem}
In this section, we derive conserved vectors for coupled KdV equations (\ref{real})-(\ref{imgry}) by a new theorem due to Ibragimov. The concepts used here are from \textbf{Section \ref{ibra} Ibragimov's conservation theorem } in Chapter \ref{ch2}.
The adjoint equations for the nonlinear system coupled KdV Equations (\ref{real})-(\ref{imgry}) are \begin{align} \label{ad1}\Delta_1^{*}& \equiv  f_t+ \alpha \ uf_x+\alpha vg_x+ \beta f_{xxx}=0,\\
\label{ad2}\Delta_2^{*}& \equiv  g_t-\alpha  vf_x+ \alpha ug_x+ \beta g_{xxx}=0.
\end{align} The formal Lagrangian $\mathcal{L}$ for the nonlinear coupled system of the KdV Equations (\ref{real})-(\ref{imgry}) and its adjoint Equations (\ref{ad1})-(\ref{ad2}) is given by \begin{equation} \mathcal{L}= f \{u_t + \alpha u u_x- \alpha v v_x + \beta u_{xxx} \}  + g \{ v_t + \alpha  u v_x + \alpha v u_x + \beta v_{xxx}\},
\end{equation} where $f$ and $g$ are new variables.
We shall use the Lie point symmetries of the system (\ref{real})-(\ref{imgry}) ,namely \begin{align}
X_1 = \partial_t, \quad 
X_2 = \partial_x, \quad 
X_3 = \alpha t \partial_x+  \partial_u,\quad 
X_4 =3t \partial_t + x \partial_x - 2 u  \partial_u-2 v  \partial_v,
\end{align} to derive conserved vectors corresponding to each symmetry below.
\begin{enumerate} [\text{Case} (i)]
\item The symmetry $ X_1 = \partial_t$ yields Lie characteristic functions given by  $	W_1^1 =-u_t$ and \newline $W_1^2 =-v_t.$ Hence the  associated conserved vector is given by \begin{align} \begin{aligned}
T_1^t=& \alpha \left[f \{ u u_x- v v_x \} + g \{v u_x + u v_x \}  \right]    + \beta  \{ f u_{xxx}+ gv_{xxx} \} ,\\
T_1^x=&  \alpha \left[ f \{-  uu_t+vv_t \} - g\{ v u_t+uv_t\}\right]  \\& +  \beta \{ f_x u_{tx} + g_x v_{tx} - u_t f_{xx} -v_t g_{xx} - fu_{txx} - gv_{txx}\}. \end{aligned}\end{align}
\item The symmetry $ X_2 = \partial_x$ yields Lie characteristic functions  $
W_2^1 =-u_x, \,\, \text{and}$ $  W_2^2 =-v_x$. Therefore the associated conserved vector is
\begin{align}
T_2^t&=- u_x f - v_xg,\\
 T_2^x&= fu_t + gv_t + \beta\{  -u_xf_{xx} -v_x g_{xx} + f_{x}u_{xx} + g_{x} v_{xx}\}. 
\end{align}
\item The symmetry $ X_3=  \alpha t \partial_x+  \partial_u$ yields Lie characteristic functions given by \newline$W_3^1 =1-\alpha t u_x, \,\, \text{and} \,\, W_3^2 =-\alpha tv_x.$
Hence the associated  conserved vector is given by \begin{align} \begin{aligned}
T_3^t= &f- \alpha t \{  u_x f +  v_x g\},\\
T_3^x=& \alpha \left[ fu+gv+ t\{ u_tf+v_tg\} + \beta t\{ \frac{f_{xx}}{ \alpha t} -u_xf_{xx} -v_x g_{xx} + f_x u_{xx} + g_x v_{xx}    \}\right]. \end{aligned}
\end{align}

\item The symmetry $ X_4 = 3t  \partial_t + x \partial_x - 2 u \partial_u-2 v  \partial_v$ yields  the Lie characteristic functions: 
$W_4^1 =-2u -3tu_t-xu_x, \quad  W_4^2 =-2v -3tv_t-xv_x.
$ Consequently, the  corresponding conserved vector is given by
\begin{align} \begin{aligned} T_4^t&=\alpha \left[ 3t \{fuu_x-fvv_x+guv_x+gvu_x \}  \right]+ \beta \left[ 3t\{ fu_{xxx} + gv_{xxx}\}\right]\\& -2\{fu+gv \}-x\{ fu_x+gv_x\},
\\
T_4^x& = x\{fu_t+gv_t \}+ \beta\left[ 3\begin{pmatrix} f_xu_x+g_xv_x+t\{ f_xu_{tx} +g_x v_{tx} \} \end{pmatrix}\right]\\& 
-\alpha \left[2 \left(f \{u^2-v^2 \}+2guv \right)+3t\begin{pmatrix} f\{ uu_t-vv_t\}+g\{vu_t+uv_t\} \end{pmatrix}\right] \\&
-\beta \left[ x \{ u_xf_{xx}+v_xg_{xx}-f_xu_{xx}-g_xv_{xx} \} +2 \{ uf_{xx} + vg_{xx}\}\right]\\&
-\beta \left[ 3t \{ f_{xx}u_t+g_{xx}v_t+ fu_{txx} + gv_{txx}\} + 4 \{ fu_{xx} + gv_{xx}\}\right].
\end{aligned}
\end{align} 
\end{enumerate}
\begin{rem}
The appearance of arbitrary functions $f(t,x)$ and $g(t,x)$ in the conserved vectors proves the existence of infinite conservation laws for coupled KdV system obtained by Ibagimov's method.
\end{rem}
\section{Concluding remarks}
In this chapter, we have studied the nonlinear coupled system of  KdV equations by use of Lie group analysis. Just like for the KdV 
Equation (\ref{kd}), we have symmetries that represent time and space translations, Galilean boost, and scaling for the nonlinear coupled KdV equation. For each symmetry, we performed a  reduction of the nonlinear coupled system.
All the group-invariant solutions describe the various states of the system.  Last but not least, an infinite number of conservation laws were derived for the system by the multiplier approach and Ibragimov's conservation theorem.

Three of the conservation laws, just like those for the KdV equation in Chapter \ref{ch3} manifest that mass, momentum, and energy are invariant quantities in the evolution of the coupled KdV system. In fact, only some of the first laws have a physical interpretation. Higher-order laws aid in understanding the qualitative properties of solutions. These conservation laws are very important in explaining the integrability of a system and the effectiveness of numerical methods used in approximating solutions. They can also be used to get exact solutions to the nonlinear coupled KdV system, which is a subject of future work.